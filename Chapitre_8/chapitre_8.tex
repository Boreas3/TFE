\chapter{Conclusion and perspectives}
\quad\ This chapter is here to summarize the content of the thesis. The purpose of the work was to provide some improvement to the existing Python computer code modeling a Brayton gas cycle. To understand how these improvement have been implemented, some theoretical contents were required to be introduced. 

In the chapters \ref{C2} and \ref{C3}, the basis about thermodynamic have been presented. One of the point covered by the third chapter was about the assessment the thermodynamic state of a fluid based on the knowledge of two independent state variable. The special case of the ideal gas has been emphasized.

Then, the chapter \ref{C4} focused on the theoretical contents about the components constituting the Brayton cycle. In particular, it has been introduced the principle of similarity for the turbomachines. The performance maps of the compressor and the turbine have been derived based on this principle. 

In the chapter \ref{C5}, a non exhaustive list of variant of the Brayton gas cycle has been proposed. Three of these variants have been analyzed. The purpose was to observe the impact of a regenerator and a bi-staged compression and expansion on the performance of the cycle. 

The chapter \ref{C6} provided a overview of how the program is structure. The different parts of the program have been described in order to have a good understanding about the logic behind the computer code. 

The chapter \ref{C7} continued what was done in chapter \ref{C6} by looking at the internal structure of the different elements constituting the model. One particular attention has been made on the interpolation and extrapolation of the compressor and turbine performance map. The algorithm used is the least-square regression using polynomial. Also, different methods have been proposed regarding the delimitation of the compressor map. 

Nevertheless, there are still area that haven't been covered in this work. The principal one is the consideration of the different transient effects. Taken into account these effects would allow to study the starting of the machine with the heating up of the heat exchanger and the acceleration of the turbomachines. 

Initially, one part of the work consisted in comparing the result from the computer code with ones that were obtained by the means of experimental campaigns. However, due to Covid-19 crisis and some issues regarding to the test bench, this part has been canceled to due the lack of data to be post-processed. Thus, the work initiated during this master thesis will be continued to provide continuously new features to the program. 

