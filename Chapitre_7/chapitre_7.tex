\graphicspath{{Chapitre_7/Images/}}
\chapter{Brayton cycle modeling}\label{C7}
\quad\, This chapter is about the description of the different part constituting the Brayton cycle model. The previous chapter described the main structure of the program. Also, it showed that the programming have been performed in such manner that all the object can individually be called by the main function of the model without too much difficulties.

When the objects have been built, two categories started to be drawn. The first category represents the object called \textbf{core objects}, are ones for which the ''position'' within the program does not change once the type of Brayton cycle configuration have been chosen. Those are, for instance, the compressor, the turbine, the combustion chamber,etc\dots

Then, there is a second category which corresponds to the \textbf{flow objects}. Those are essentially the different fluids that will move through the cycle. These fluids will be submitted to various transformations induced by the core objects. The flow objects are mainly composed of the composition of the represented fluids. Also, depending on the fluids, some specific procedures are embedded in these objects.

This chapter dedicated to the modeling of the Brayton cycle will describe the structure of these different objects. The methods of implementation of the theoretical notions through numerical tools will be explained as well.

\section{Flow objects}
\quad\ The first category to be considered in this chapter is the one represented the flow objects. Those, as it has been said, contain the information regarding the different fluids used within the cycle.

\subsection{Composition of the fluids}
Within a Brayton cycle, the working fluids are in the majority of the cases ar air, gaseous fuel, and exhaust gas. These two fluids can be decomposed, without losing in accuracy, into ideal gases. Those gases are ones that follows the ideal equation seen in chapter \ref{C2}. 

From common knowledge, it is known that the atmospheric air is mainly composed of 79\% of $\ce{N2}$ and 21\% of $\ce{O2}$, two gases that with a behavior closed from the behavior of an ideal gas.

Considering now the fuel, its composition really varies depending on the location. Indeed, if the fuel used in the system is natural gas, the principal component is $\ce{CH4}$, but there are also $\ce{C}_\text{m}\ce{H}_\text{n}$, $\ce{CO2}$, $\ce{N2}$,etc\dots

The Table \ref{tab:C7_compgas} gives some data about the natural gas composition for some sites.


\begin{table}[h]
\centering
\begin{tabular}{ll|lllll}
                                &                                       & \multicolumn{4}{c}{Molar fraction (in \%)}                            &                 \\ \hline
Combustible                  & Location                              & $\ce{CH4}$ & $\ce{C}_\text{m}\ce{H}_\text{n}$ & $\ce{CO2}$ & $\ce{N2}$ & $HCV_l$ (kJ/kg) \\ \hline
\multirow{4}{*}{Natural gas} & Slochteren (Netherlands)              & 81.4       & 3.5                              & 0.9        & 14.2      & 38100           \\
                                & North sea                             & 88.6       & 6.1                              & 1.4        & 3.9       & 44690           \\
                                & CIS & 92.3       & 4.3                              & 0.4        & 3.0       & 46540           \\
                                & Algeria                               & 87.0       & 12.6                             & -          & 0.4       & 49150          
\end{tabular}
\caption{Composition and lower heating calorific value of the natural gas \cite{Leonard2018}.}
\label{tab:C7_compgas}
\end{table}
with CIS being the ''Commonwealth of Independent States'' \cite{EncyclopaediaBritannica2018}.

As it can be noticed, the location of the fuel sink have a strong influence on the gas composition. Therefore, to assure the consumer that the sold gas always has the same heating calorific value, a mixture is made at the factory. 

The different components of the fuel can all be considered as ideal gases. 

Finally, there is the exhaust gas for which the composition depends on the one of the two previously mentioned gases. Indeed, the theoretical part in chapter \ref{C4} about the combustion shows that the exhaust gas composition can be obtained through one of the reactions (\ref{eq:C7_chemgeng01}).
\begin{equation}
    \setstretch{1}
    \begin{cases}
        \ce{C_{\text{m}}H_{\text{n}}O_{\text{x}}N_{\text{y}} +}\kappa\lambda \left(\ce{O2}+\frac{79}{21}\ce{N2}\right) \ce{-> mCO2 +} \kappa(\lambda-1)\ce{O2 + \frac{n}{2}H2O +} (\kappa\lambda\frac{79}{21} + \frac{\text{y}}{2})\ce{N2} & \text{ for \(\lambda\geq 1\)} \\
        \ce{C_{\text{m}}H_{\text{n}}O_{\text{x}}N_{\text{y}} +}\kappa\lambda \left(\ce{O2}+\frac{79}{21}\ce{N2}\right) \ce{-> aCO2 + bCO + \frac{n}{2}H2O} + (\kappa\lambda\frac{79}{21} + \frac{\text{y}}{2})\ce{N2}                      & \text{ for \(\lambda< 1\)}
    \end{cases}\label{eq:C7_chemgeng01}
\end{equation}
where the coefficients ''m'', ''n'', ''x'', and ''y'' have to be determined. As a reminder, $\kappa$ is a coefficient equal to m+$\frac{\text{n}}{4}$-$\frac{\text{x}}{2}$. 
\newpage
\subsubsection{Computation of the fuel coefficients}
\quad\ In the equations, the used fictitious fuel is $\ce{C_{\text{m}}H_{\text{n}}O_{\text{x}}N_{\text{y}}}$. The coefficients ''m'', ''n'', ''x'' and ''y'' are obtained by analyzing the composition of the real fuel.

To determine these coefficients, the knowledge of the molar fraction $x_i$\footnote{with $i$ being a given species} of the different species is required. If the fuel composition is given as mass fraction, the conversion is made using the formula (\ref{eq:C7_y2x}).

\begin{equation}
\setstretch{1}
    x_i = y_i\cdot \frac{MM_{tot}}{MM_i}\label{eq:C7_y2x}
\end{equation}

with $MM_i$ and $MM_{tot}$ being respectively the molar mass of the specie $i$ and of the fuel.
Then, a filter is applied to eliminate the species that will not react during the combustion. Among these components considered as stable, there are the water \ce{H2O}, the nitrogen \ce{N2} and the carbon dioxide \ce{CO2}. Other elements, like the noble gases, could be considered as well but those are rarely present within a fossil fuel.

Once the filter is done, the global mass fraction of the remaining part of the fuel, called fictitious fuel, is computed. The expression of the real fuel composition is then  ($y_{\ce{C_{\text{m}}H_{\text{n}}O_{\text{x}}N_{\text{y}}}}$, $y_{\ce{H2O}}$, $y_{\ce{N2}}$, $y_{\ce{CO2}}$).

Considering now only the fictitious fuel $\ce{C_{\text{m}}H_{\text{n}}O_{\text{x}}N_{\text{y}}}$, the fuel coefficients can be obtained. This is done by iterating over the different species contained within the fictitious fuel. Initially, the coefficients ''m'' to ''y'' are set to 0. The following operation are then performed for each species constituting the fictitious fuel:
\begin{itemize}
    \item First, the program analyze the atomic composition of the component. It counts, for one mole, the number of moles of carbon \ce{C}, hydrogen \ce{H}, oxygen \ce{O}, and nitrogen \ce{N} that composed the component. For instance, the returning result for \ce{CH4} will be ($x_{\ce{C}}$, $x_{\ce{H}}$, $x_{\ce{O}}$, $x_{\ce{N}}$) = (1, 4, 0, 0).
    \item Then, the contribution of the component to the fuel coefficients is taken into account. By considering the molar fraction $x_{\left.i\right|_f}$ of the component in the fictitious fuel, the four coefficients are updated according to the rules (\ref{eq:C7_updatecoef}).
    
    \begin{equation}
    \setstretch{1}
        \begin{cases}
        \text{m}=\text{m}+x_{\left.i\right|_f}\cdot x_{\ce{C}}\\
        \text{n}=\text{n}+x_{\left.i\right|_f}\cdot x_{\ce{H}}\\
        \text{x}=\text{x}+x_{\left.i\right|_f}\cdot x_{\ce{O}}\\
        \text{y}=\text{y}+x_{\left.i\right|_f}\cdot x_{\ce{N}}
        \end{cases}\label{eq:C7_updatecoef}
    \end{equation}
\end{itemize}
After having perform this iterative process, the computed coefficients are stored in memory to be used during the fumes composition computations.
\subsubsection{Exhaust gas compositions}
\quad\ The preliminary computations described in the previous lines were required to compute the coefficients ''m'', ''n'', ''x'', and ''y'' of the fictitious fuel $\ce{C_{\text{m}}H_{\text{n}}O_{\text{x}}N_{\text{y}}}$. From this point, the composition of the fumes can easily be obtained. 

Indeed, by performing a quick analysis of the first reaction of (\ref{eq:C7_chemgeng01}), the relations (\ref{eq:C7_O2}) to (\ref{eq:C7_CO2y}) can be obtained.
\begin{subequations}
\begin{align}
    w_{\left.\ce{O2}\right|{fumes}} &= w_{\left.\ce{O2}\right|{air}} - \kappa\cdot w_{\left.\ce{C_{\text{m}}H_{\text{n}}O_{\text{x}}N_{\text{y}}}\right|{fuel}}\label{eq:C7_O2}\\
    \rightarrow & y_{\left.\ce{O2}\right|{fumes}} =  w_{\left.\ce{O2}\right|{fumes}}\cdot \frac{MM_{\ce{O2}}}{\dot{m}_{gas}}\label{eq:C7_O2y}\\
    w_{\left.\ce{N2}\right|{fumes}} &= w_{\left.\ce{N2}\right|{air}} + w_{\left.\ce{N2}\right|{fuel}} + \frac{\text{y}}{2}\cdot w_{\left.\ce{C_{\text{m}}H_{\text{n}}O_{\text{x}}N_{\text{y}}}\right|{fuel}}\label{eq:C7_N2}\\
    \rightarrow & y_{\left.\ce{N2}\right|{fumes}} =  w_{\left.\ce{N2}\right|{fumes}}\cdot \frac{MM_{\ce{N2}}}{\dot{m}_{gas}}\label{eq:C7_N2y}\\
    w_{\left.\ce{H2O}\right|{fumes}} &= w_{\left.\ce{H2O}\right|{air}} + w_{\left.\ce{H2O}\right|{fuel}} + \frac{\text{n}}{2}\cdot w_{\left.\ce{C_{\text{m}}H_{\text{n}}O_{\text{x}}N_{\text{y}}}\right|{fuel}}\label{eq:C7_H2O}\\
    \rightarrow & y_{\left.\ce{H2O}\right|{fumes}} =  w_{\left.\ce{H2O}\right|{fumes}}\cdot \frac{MM_{\ce{H2O}}}{\dot{m}_{gas}}\label{eq:C7_H2Oy}\\
    w_{\left.\ce{CO2}\right|{fumes}} &= w_{\left.\ce{CO2}\right|{air}} + w_{\left.\ce{CO2}\right|{fuel}} + \text{m}\cdot w_{\left.\ce{C_{\text{m}}H_{\text{n}}O_{\text{x}}N_{\text{y}}}\right|{fuel}}\label{eq:C7_CO2}\\
    \rightarrow & y_{\left.\ce{CO2}\right|{fumes}} =  w_{\left.\ce{CO2}\right|{fumes}}\cdot \frac{MM_{\ce{CO2}}}{\dot{m}_{gas}}\label{eq:C7_CO2y}
\end{align}\label{eq:C7_compfumes}
\end{subequations}
Where $w_{\left.i\right|fluid}$ corresponds to the molar flow rate of the element $i$ within the $fluid$. The mass flow rate $\dot{m}_{gas}$ is obtained as being the sum $\dot{m}_{air}+\dot{m}_{fuel}$.

\subsubsection{Liquid composition}
\quad\ Aside the gases, water will also be used when considering the water heat-exchanger. For this particular case, it will be supposed that pure water is used.

\subsection{Thermodynamic state assessment}
\quad\ A method to compute the composition of the exhaust gas based on the air and fuel composition has been established in the previous section. Now that the required tools to determine the fluid compositions have been developed, it is possible to evaluate the state of the fluids. 

To minimize the computation time, only the state at the beginning and ending of each core objects will be evaluated. The reason behind this choice is that it is sufficient to know the state at the inlet and outlet of each components to assess the power production/consumption, the heat transfer between two fluids,etc\dots

Assuming that the temperature and pressure at a given point are known, there exist several possibilities regarding to the evaluation of the flow state at this point. 

The first option consists in using CoolProp \cite{Bell2014}, an open-source thermodynamic library. This tools, as it has been mentioned in the chapter \ref{C3}, uses the Helmotz and Gibbs functions to evaluate the desired state variables. When using CoolProp, the state is accurately assess since the library is solving the exact partial derivative equations. 

Now, when considering an ideal, there exists an alternative to CoolProp for the calculation of the enthalpy and the heat capacity. In the chapter \ref{C3}, it has been shown that those quantity can only be computed using the temperature. This valuable property could be used to build polynomial of the type f(T) to estimate the value of the enthalpy and heat capacity for a given temperature T. 

Since the middle of the last century, scientists started to conduct experiments to assess the state of various gases to a large range of temperature. For each temperature tested, records have been made to build a table. This table, called thermodynamic table, is composed of rows corresponding each to one temperature value for which states have been recorded. 
An example of thermodynamic table is given in Table \ref{tab:C7_thermotab}.
Among the differents 