\chapter{Notions of thermodynamics}
\quad\, In the introduction, it has been introduced many times the concept and the used of the Brayton cycle. As explained, this concept has been applied for many usages, including electricity production and aircraft propulsion which are the most common and known applications. It has been exposed that from the invention proposed by George Brayton in the end of the 19th century, the technology did really evolve.

Indeed, while the Brayton cycle engine first appears as a  piston engine where the compression, combustion and expansion occurs in the same enclosure, Nowadays 
the process is shared between at least three components (namely the compressor, the turbine and the combustion chamber).

However, what the previous did not cover were the keys notions to understand how theses components behave. Those notions, which will be used during this work, need to be defined and explained to allows to the common person to understand this writing without possessing those knowledge a priory. 
This problematic will be covered by this chapter, which will introduce step-by-step those notions.

\section{Fundamental notions}
\quad\, As mention in the lead-in of this chapter, the first sections of this report will entirely be devoted to the bringing in of the required knowledge for the understanding of the full report.

Starting from very fundamental notions, those will allow to explain more complex concept that will be applied in this work.
\newpage
\subsection{Open/closed system}\label{sect:C2_Sys}
\quad\,  The thermodynamic is a science that "studies the exchange of energy between a system and its environment or surrounding" \cite{thermoApp_1}.

The system is defined as being the area of the space selected for the study. Between the system and the environment lies the boundary. This boundary can either be real or fictitious and, can be static or mobile.

When the system is characterised, it has to be established if it is an open or a closed system.
The open systems are ones where an arbitrary control volume well demarcated in the space is studied. For these systems, matter and energy is exchanged with the surrounding as heat or work. Typical examples are combustion chamber, heat-exchanger, turbomachines, piping,...

On the other hand, the closed systems does not exchange matter with the environment. Indeed, 
for those systems a control mass well delimited in the space is studied. Therefore, the exchange of mass with the environment is prohibited for this category of system. 

\subsection{State functions and variables}\label{sect:C2_State}
\quad\, Let considered a system as defined in the previous section. The state functions are defined as "a property of the system that only depend on its current state". These functions are independent of the past of the system and describe the equilibrium state of the system.

When the state functions can be measured (directly or indirectly), these are called state variables. For example, the pressure $P$, the temperature $T$ or the volume $V$ are state variables.

It can be shown that the equilibrium state of a system can be described by those three variables. Also, among $P$, $T$ and $V$, only two of them are independent. This means that a state relation defined as \textbf{F}($P$, $T$, $V$) = 0.

For the ideal gas, this relation is
\begin{equation}
P\cdot V = m\frac{R}{MM}\cdot T\label{eq:C2_GP}    
\end{equation}
where \textit{m} is the quantity of matter (in kg), $R$ is the universal gas constant (8.314 J/mole/K\footnote{K is degree Kelvin and is scaled as follows: 273.15\degree K = 0\degree C}) and $MM$ is the molar mass (in kg/mole) of the system.
\newpage
\subsection{Energy}\label{sect:C2_Ener}
\quad\, In the section \ref{sect:C2_Sys}, the notion of energy has been mentioned without characterising it before hand. From the first law of thermodynamic, it is stated that the energy cannot be created nor destroyed. Therefore, it can only be converted an can exist into multiple forms (thermal, mechanical, electrical,...)\cite{thermoApp_2}. The unit of the energy is the Joule (J), and the sum of all the energies in a system is called the total energy.

This total energy always depends on a reference point because an absolute value for the energy is not defined. Usually, the reference point is set to 0.

As said in the section \ref{sect:C2_Sys}, any system will transfer its energy to the surrounding as heat and/or work. The heat is defined as "the form of energy which is exchanged between the system and its environment when there is a gradient (=difference) of temperature between these two entities". As the heat is always referred as a flux between two bodies, the phenomena is called heat transfer. The heat transfer can be realised by
\begin{itemize}
    \item Conduction: The heat is transferred through a non flowing material due to the interaction of "molecular scale energy carriers" inside the material
\end{itemize}
