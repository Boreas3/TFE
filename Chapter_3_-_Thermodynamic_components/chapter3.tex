\graphicspath{{Chapter_3_-_Thermodynamic_components/Images/}}
\chapter{Thermodynamic components}
\quad\, In the beginning of the previous chapter, it has been mention that the Brayton cycle composed of several components that are more or less complex. The behavior of these components, which is required to realized the study the global system, is based on the thermodynamic notions that have be introduce all along the past lines.

This chapter will be focused on the description of those components. For each of them, it will be provided a state of the art followed by the concepts or principles that will be used during this work. 
\section{Turbomachines}
\quad\, The first family of components to be studied is the turbomachines. The machines owning to this family are ones "that exchange energy between the
fluid traversing it and mechanical energy supplied to or extracted from the machine" \cite{Hillewaert2019}. Those machines are \textbf{rotating} machines and, based on the compressible nature of the fluid, two categories can be created.
This subsection will swept the two categories but will mainly focused on the machines exchanging energy with \textbf{compressible} flow.

\subsection{Incompressible flow}
\quad\, The first category of flow to be considered is the incompressible flow. This type of flow is characterized by a constant density over the distance. An example of incompressible fluid is the water.

Among the machine exchanging energy with such type of fluid, there are pumps which are designed to raise the height or total hydraulic energy $h$ of the fluid. The variation of the height of the fluid is similar to its enthalpy variation. Thus, the power output  $\dot{W}_p$ developed by the pump can be expressed as given in relation (\ref{eq:C3_Ppump})
\begin{equation}
\dot{W}_{p,a-b} = \dot{m}\cdot (h_a - h_b)=\dot{m}\cdot\Delta h_p \label{eq:C3_Ppump}
\end{equation}
considering that the transformation makes the system going from state \textbf{a} to state \textbf{b}.

If the consumed power of the pump is $\dot{W}_{e,a-b}$, its global efficiency $\eta_p$ is equal to the ratio
\begin{equation}
\eta_p = \frac{\dot{W}_{p,a-b}}{\dot{W}_{e,a-b}}\label{eq:C3_Etapump}
\end{equation}
\subsubsection{Characteristic maps}
\quad\, It has be shown that the power output and the global efficiency of the pump are functions of the height variation and the flow rate of the fluid. When operating a pump, it can be useful to know how this height variation will vary with respect to the flow rate and the rotational speed of the pump shaft. The knowledge of these two parameters allows to fully characterized the pump.

Considering the volumetric flow rate $Q_p$ (m$^3$/s) and the rotational speed $N$, the two following relations (\ref{eq:C3_DHp}) and (\ref{eq:C3_Pe}) can be derived.

\begin{subequations}
\setstretch{1}
\begin{equation}
\Delta h_p = \Delta h_p(Q_p, N)\label{eq:C3_DHp}\\
\end{equation}
\begin{equation}
\dot{W}_e = \dot{W}_ef(Q_p, N)\label{eq:C3_Pe}
\end{equation}
\end{subequations}
Those relations will be called characteristic or performance Map and are determined \textbf{experimentally}. An example of such map is given on Figure \ref{fig:C3_MapPump}.
\begin{figure}[h]
\centering
\includegraphics[width=0.8\textwidth]{char_map_pump.png}
\caption{Characteristic maps of a pump \citep{Hillewaert2019}}
\label{fig:C3_MapPump}
\end{figure}
\subsubsection{Similarity}
\quad\, When considering incompressible flow, it is possible to extrapolate from a known operating point a infinity of similar operating points. Indeed, There exist relationships which provides with enough accuracy the change of flow rate $Q$ and height variation $\Delta H$ when the rotational speed goes from $N_1$ to $N_2$. It can be demonstrate that the flow rate evolves linearly with the rotation speed, and that the height variation is a quadratic function of $N$.

\begin{subequations}
\setstretch{1}
\begin{equation}
Q_2 = Q_1\cdot\frac{N_2}{N_1} \label{eq:C3_Qsim}
\end{equation}
\begin{equation}
\Delta h_2 = \Delta h_1\cdot\left(\frac{N_2}{N_1}\right)^2 \label{eq:C3_DHsim}
\end{equation}\label{eq:C3_sim}
\end{subequations} 

These relations are really useful to extrapolate the performance maps of a pump. Similar relations can be deduced considering the variation of the radius of the pump.

One important property to notice is that all the dimensionless variables (e.g. the efficiency) are kept constant for all the different similar operational points. This is a valuable property which will be very useful for the future developments of the thesis.
\subsubsection{Types of pumps}
\quad\, Now that the exterior characteristics of the pumps have been defined, it is interesting to have at least a brief idea about how the pump is constructed. Without entering into detailed\footnote{see the section 4.3 and 4.4 of the course \citep{Hillewaert2019}}, there are two types of pumps.

The first type to be considered is the centrifugal pumps designed to provide a high heat for a low flow rate. Those pumps are characterized by an axial inflow and a radial outflow. The Figure \ref{fig:C3_centri_pump} shows a schematic of a centrifugal pumps.
\begin{figure}[h]
\centering
\includegraphics[width=0.4\textwidth]{centri_pump.png}
\caption{Centrifugal pump \citep{Hillewaert2019}}
\label{fig:C3_centri_pump}
\end{figure}

Basically, the centrifugal pump can be decomposed into two parts (for the most simple device). The first  part is the rotating impeller that will convert and transfer the mechanical energy to the fluid. Behind the impeller will be placed the volute (right picture of Figure \ref{fig:C3_centri_pump}) that collects the flow to bring it to the outlet of the pump.\newpage

The second type of pump are the axial pumps which are, in opposition with the centrifugal pumps, designed to deliver low head for high flow rates. Such pumps is illustrated on Figure \ref{fig:C3_axial_pump}. 
\begin{figure}[h!]
\centering
\includegraphics[width=0.25\textwidth]{axial_pump.png}
\caption{Axial pump \citep{Hillewaert2019}}
\label{fig:C3_axial_pump}
\end{figure}

The main two parts of an axial pumps are the rotor (the rotating part) that will increase the height of the fluid followed by a diffusor which "recuperates the kinetic energy at the exit of the rotor"\citep{Hillewaert2019}.

\subsection{Compressible flow}
\quad\, The previous subsection introduced the pump which is a turbomachine design to increase the energy of the incompressible fluid passing through it. However, this type of machine cannot deals with compressible flow for which the density can vary over the distance. For instance, the air is a compressible fluid. 

The behavior of the compressible flow is more complex to describe compared to incompressible flow. Indeed, "compressible flow is characterized by the propagation of acoustic waves"\citep{Hillewaert2019}.   This part of the section about turbomachines will only focused on the very main principles required for the good understanding of this work.

\subsubsection{Static and total quantities}
\quad\, As it has be partly reveal in the previous lines, the flow characteristics are dependent on its velocity. The study of compressible flow did lead to the distinction between the static and total quantities. 

The static quantities are state variables (e.g. temperature, pressure,...) that are independent of the flow velocity. As for the total quantities, these are dependent of the flow velocity $V$. For instance, the total enthalpy is given by
\begin{equation}
h^0 = h + \frac{1}{2}\cdot v^2\label{eq:C3_h0}
\end{equation}
where the total quantity is identified by the superscript "0".

The total enthalpy can be defined as "the static enthalpy obtained when the gas is brought adiabatically to a halt"\citep{Hillewaert2019}.


\subsubsection{Conservation of the total enthalpy and rothalpy}
\quad\, For an adiabatic transformation without viscous work, the total enthalpy is conserved between the initial state and the final state.
\begin{equation}
\dot{m}\cdot h_1^0 = \dot{m} h_2^0 \label{eq:C3_hcons}
\end{equation}
which can be reduced to $h_1^0 = h_2^0$ if we supposed that the transformation is performed without any leakages. The states $1$ and $2$ are associated to the orthogonal boundaries to the flow of the selected control volume delimiting the studied system. 

Now, considering a rotating system (e.g. a rotor), the variation of the total enthalpy can be obtained by considering the modified Euler equation of turbomachinery
\begin{align}
\setstretch{1}
h_2^0 - h_1^0 = \frac{1}{2}\cdot &\left(v_2^2 - v_1^2\right) - \frac{1}{2}\cdot \left(wr_2^2 - wr_1^2\right) + \frac{1}{2}\cdot \left(ur_2^2 - ur_1^2\right)\label{eq:C3_Euler}\\
\text{with }& v = \left|\vect{v}\right|\quad\text{;}\quad  wr = \left|\vect{wr}\right|\quad\text{;}\quad ur= \left|\vect{ur}\right|\nonumber
\end{align}
where $\vect{ur}$, $\vect{v}$ and $\vect{wr}$ are respectively the local velocity of the rotor, and the absolute and relative velocity (with respect to the rotor) of the flow. Defining the total rothalpy as being 
\begin{equation}
i^0 = h + \frac{1}{2}\cdot wr^2 - \frac{1}{2}\cdot ur^2 \label{eq:C3_i0}
\end{equation}
It is obtained from the Euler equation (\ref{eq:C3_Euler}) that the total rothalpy is conserved through the transformation.
\begin{equation}
i_1^0 = i_2^0 \label{eq:C3_icons}
\end{equation}
\subsubsection{Velocity triangle}
\quad\, With the relation (\ref{eq:C3_Euler}), the notion of absolute and relative velocity of the flow has been introduced. A graphic representation of these three vector can be done using the \textbf{velocity triangle}. This triangle is drawn on Figure \ref{fig:C3_vtriang}.
\begin{figure}[h]
\centering
\includegraphics[width=0.6\textwidth]{Vtriangle.png}
\caption{Velocity triangle}
\label{fig:C3_vtriang}
\end{figure}

\subsubsection{Mach number}
The Mach number $M$ is defined as being the ratio between the velocity $v$ and the sound speed $a$.
\begin{equation}
M = \frac{v}{a} \label{eq:C3_Mach}
\end{equation}
The Mach number $M$ is a dimensionless variable that gives an image of the compressible effects of the flow. Thus, one criteria for the determination of similar operational points is to keep constant the Mach number.

Using the Mach number allows to obtain formulas to compute the total quantities base the static ones. By considering first the total temperature, it can be found
\begin{equation}
T^0 = T + \frac{v^2}{2\cdot c_p} = T\cdot\left(1 + \frac{v^2}{2\cdot c_p\cdot T}\right)\label{eq:C3_TT0_1}
\end{equation}
For an isentropic process, it can be demonstrate that the speed of sound $a=k\cdot r\cdot T$. Thus, the equation (\ref{eq:C3_TT0_1}) becomes
\begin{equation}
T^0 = T\cdot\left(1 + \frac{k-1}{2}\cdot M^2\right) = T\cdot f(M) \label{eq:C3_TT0}
\end{equation}
Using the equations (\ref{eq:C2_isrelPT}), (\ref{eq:C2_isrelrhoT}) and the definition of the function $f(M)$, the relations linking the static to the total pressure, density and speed of sound can be obtained as well.

\begin{subequations}
\setstretch{1}
\begin{equation}
p^0 = p\cdot f(M)^\frac{k}{k-1}\label{eq:C3_PP0}
\end{equation}
\begin{equation}
\rho^0 = \rho\cdot f(M)^\frac{1}{k-1}\label{eq:C3_rhorho0}
\end{equation}
\begin{equation}
a^0 = a\sqrt{f(M)} \label{eq:C3_aa0}
\end{equation}
\end{subequations}

\subsubsection{Characteristic maps}
\quad\, As for the turbomachines exchanging energy with incompressible flow, those dealing with compressible flow can also be fully characterized knowing a pair of independent operating parameters. The most usual parameters are

\begin{itemize}
\setstretch{1}
\item $\dot{m}_c$ (kg/s or lbs/min): It is the corrected mass flow rate.

\item $N$ (rpm): It is the rotational speed of the turbomachine shaft. 

\item $\Pi$ (-): It is the pressure ratio between the inlet and the outlet of the turbomachines. If the turbomachines is compressing the flow, the ratio is reverse to keep it greater than one.

\item $\eta$ (-): It is the isentropic efficiency of the machine.
\end{itemize} 

The corrected mass flow rate is defined as follows

\begin{equation}
\dot{m}_c = \dot{m}\cdot \sqrt{\frac{T}{T_{ref}}}\cdot\left(\frac{p_{ref}}{p}\right)
\end{equation}

with $T_{ref}$ and $p_{ref}$ being the reference temperature and pressure (all quantities being static).

Let's supposed now that the rotational speed and the corrected mass flow are known. Then, the two other quantities ($\Pi$ and $\eta$) can be calculated by evaluating the relationships (\ref{eq:C3_Pimap}) and (\ref{eq:C3_etamap}).

\begin{subequations}
\setstretch{1}
\begin{equation}
\Pi = \Pi(N, \dot{m}_c)\label{eq:C3_Pimap}
\end{equation}
\begin{equation}
\eta = \eta(N, \dot{m}_c)\label{eq:C3_etamap}
\end{equation}
\end{subequations}  
These relations can be established by the means of experimental measurements.
\subsection{Gas compressor}
\quad\, The previous lines shows some properties of compressible flows that add complexity to the analysis. Now, it is interesting to describe two types of turbomachines inducing a modification of the state of the flow. 

The first machines to be describe are the compressor. As for the pumps, the compressor can be axial or centrifugal.  The axial compressor are mainly composed of a rotating part (the rotor), followed by a non moving part (the stator) converting the the kinetic energy at exit of the rotor into pressure. A schematic of an axial compressor stage is given on Figure \ref{fig:C3_compstage}.

\begin{figure}[h]
\centering
\includegraphics[width=0.5\textwidth]{Comp_stage.png}
\caption{Axial compressor stage \citep{Hillewaert2019}}
\label{fig:C3_compstage}
\end{figure}

The half-moon entities on the graph represent the blades of the rotor and stator (respectively the left and the right ones). It can be observed that the rotor blades increase the value of the flow velocity $v$. As explained earlier this augmentation of kinetic energy is then recovered by the stator blades. 

\subsubsection{Mollier diagram}
\quad\, The transformation induced by the compressor can be represented into a hs diagram also named \textbf{Mollier diagram}. The Mollier diagram for one compressor stage has been drawn on Figure \ref{fig:C3_Molliercomp}.

\begin{figure}[h]
\centering
\includegraphics[width=0.5\textwidth]{Comp_mollier.png}
\caption{Mollier diagram of a compressor stage \citep{Hillewaert2019}}
\label{fig:C3_Molliercomp}
\end{figure}

The diagram, for which the velocity triangles are given on Figure \ref{fig:C3_compstage}, has to be read as follows.
\begin{itemize}
\setstretch{1}
\item State \textbf{1}: Starting from the static enthalpy and pressure $h_1$ and $p_1$, the relations (\ref{eq:C3_i0}), (\ref{eq:C3_h0}) and (\ref{eq:C3_PP0}) are used to calculate the total enthalpy, rothalpy and pressure.  
\item State \textbf{2}: The transformation \textbf{1}-\textbf{2} takes place within the rotor of the compressor stage. Thus, the total rothalpy is conserved over the transformation ($i_2^0=i_1^0$). 

The knowledge of the total rothalpy allows to determine the other quantities. The static enthalpy (and by extension the total enthalpy) can be deduced using the relation (\ref{eq:C3_i0}). Concerning the pressure, there is an increase of the static and total pressure of the fluid "due to the diffusion in the rotor passage"\citep{Hillewaert2019} ($p_2^{(0)} > p_1^{(0)}$).
 
\item State \textbf{3}: The transformation \textbf{2}-\textbf{3} takes place within the stator of the compressor stage. Thus, the total enthalpy is conserved over the transformation ($h_3^0 = h_2^0$).

Then, the static enthalpy can be computed using the relation (\ref{eq:C3_h0}). The conversion of the kinetic energy into pressure leads to an increase of the static and total pressure ($p_3^{(0)} > p_2^{(0)}$).
\end{itemize}
As it was mentioned in the subsection \ref{C2:Isen_eff}, the enthalpy at the end of the process is higher than the one considering an isentropic transformation.
\subsubsection{Performance maps}
\quad\, As for any turbomachines, the compressor can be characterized by its performance map. The map is composed of the performance plot often completed with the efficiency hill. The performance plot provides the total to total (TT) pressure ratio $\Pi_c$ as a function of the rotational speed $N$ and the corrected flow rate $\dot{m}_c$.

A illustration of a compressor performance map is given on Figure \ref{fig:C3_compmap}.
\begin{figure}[h]
\centering
\includegraphics[width=0.6\textwidth]{Comp_Map.png}
\caption{Illustration of a compressor performance map \citep{Ghorbanian2009}}
\label{fig:C3_compmap}
\end{figure}  

On the map, the solid and the dash curves represent respectively the iso-rotational speed and the iso-efficiency.

To remarkable lines are emphasized on the plot. For each rotational speed $N_c$, these lines provides minimal and maximal bound for the compressor pressure ratio $\Pi_c$. 

The lower limit is given by the choke line. "When the flow reaches the velocity at some cross-section"\citep{Ghorbanian2009}, an increase of the gas mass flow rate is not possible. Thus, the slope of the associated iso-rotational speed becomes equal to the infinity at this point. 

The upper bound is given by the "aerodynamic stability limit line" or surge line. At this minimal flow rate, the flow start to stall in the machine. Going beyond this limit would lead to a backward flow in the compressor, causing irreversible damages to the compressor. Thus when operating the machine, a certain margin is taken with respect to the surge line\footnote{For more information about the stall and surge events, sees section 9.5 of the course "Turbomachines" \citep{Hillewaert2019}}.

The compressor map, for the low rotational speeds, can be extrapolated by considering similar operational points. Indeed, for these low speed the approximation of considering an incompressible flow instead of a compressible is accurate enough. Though, the compressor can be assimilated to a pump and, using the relations of similarity \label{eq:C3_sim}, the extrapolation is possible. 

\subsection{Gas turbine}
\quad\, From now, only compressors have been introduced. However, in a gas cycle, the compressor are combined with a turbine that will expand the gas. Only for some exception, the turbine is always placed after the combustion chamber. This choice allows to provide to the turbine a flow with very high enthalpy.

Typically, an axial gas turbine is composed of a stator followed by a rotor. The stator is design to create a deflection of the flow in the sens of rotation of the rotor. This will accelerate the flow before entering the rotor.

The schematic of an axial turbine stage is drawn on Figure \ref{fig:C3_turbstage}.
\begin{figure}[h]
\centering
\includegraphics[width=0.5\textwidth]{Turb_stage.png}
\caption{Axial turbine stage \citep{Hillewaert2019}}
\label{fig:C3_turbstage}
\end{figure}

\subsubsection{Mollier diagram}
As for the compressor, the real expansion induced by the turbine can be represented into a Mollier diagram. The diagram is depicted on Figure \ref{fig:C3_Mollierturb}. The methodology to read the diagram is the same as before.


\begin{itemize}
\setstretch{1}
\item State \textbf{1}: Starting from the static enthalpy and pressure, the total quantities can be deduced.
\item State \textbf{2}: The transformation \textbf{1}-\textbf{2} takes place within the stator of the turbine stage. Therefore, the total enthalpy is conserved over the transformation ($h_2^0=h_1^0$). 

Knowing the total enthalpy of the state \textbf{2} allows to compute the static enthalpy and by extension the total rothalpy. Plus, since the flow is accelerated by the stator blades, the static and total pressure falls from state \textbf{1} to state \textbf{2} ($p_2^{(0)}<p_1^{(0)}$).
\item State \textbf{3}: The transformation \textbf{2}-\textbf{3} takes place within the rotor of the turbine stage. Therefore, the total rothalpy is conserved over the transformation ($i_3^0=i_2^0$).

Then, the relation (\ref{eq:C3_i0}) allows to retrieve the static enthalpy of state \textbf{3}. As shown on the schematic \ref{fig:C3_turbstage} of the turbine stage, the relative speed of the flow is augmented when going through the rotor passage. This generates a second drop of the static and total enthalpy and pressure ($p_3^{(0)}<p_2^{(0)}$). 

However, there exists special stage for which "the relative velocity of the flow to the blade row" of the rotor\citep{Hillewaert2019} remains unchanged, but the flow is inverted. Neglecting the very small variation of the rotor velocity, the static enthalpy and pressure remains unchanged when passing through the rotor.
\end{itemize} 

\begin{figure}[h]
\centering
\includegraphics[width=0.5\textwidth]{Turb_mollier.png}
\caption{Mollier diagram of a turbine stage \citep{Hillewaert2019}}
\label{fig:C3_Mollierturb}
\end{figure}

\subsubsection{Axial turbine design and degree of reaction}
\quad\, The last written paragraph mentioned that some turbine are designed to put all the pressure drop over the stator to limit the axial force generated in the rotor.

It is possible to classify the turbines by creating two categories based on the degree of reaction $R$. The degree of reaction is defined as being the ratio between the static enthalpy drop over the rotor and the static enthalpy drop over the turbine stage.
\begin{equation}
R = \frac{h_2 - h_3}{h_1 - h_3}\label{eq:C3_R}
\end{equation}
Then the two families are 

\begin{itemize}
\setstretch{1}
\item The impulse or action turbines: The degree of reaction $R\simeq 0$. Thus, since the rotor doesn't see any pressure difference, the axial thrust on the shaft is minimized. The impulse turbines are often used for high pressure turbines (due to their small aspect ratio).
\item The reaction turbines: The degree of reaction $R$ is often between 0.5 and 0.7. This means that the pressure drop is distributed between the stator and the rotor. 
\end{itemize}

\subsubsection{Performance maps}
\quad\, The turbines can be also be characterized by a performance map determined from experimental results. The map is often composed of two performance plots as shown on Figure \ref{fig:C3_turbmap}.

\begin{figure}[h]
\centering
\includegraphics[width=\textwidth]{Turb_Map.png}
\caption{Illustration of a turbine performance map}
\label{fig:C3_turbmap}
\end{figure}  

The two plots on the Figure \ref{fig:C3_turbmap} illustrates for each iso-rotational speed $N_t$ the relationships $\Pi_t(\dot{m}_c,N_t)$ and $\eta_t(\dot{m}_c,N_t$. As it can be noticed on the left diagram, all the curves tends to reach the same asymptotic values as the mass flow rate increases. This is due to the choking phenomena that limits at some point the flow rate through the compressor.\\

In this section about turbomachines, the most important notions for the good understanding of the future development have been given. Particularly, the concept of performance maps has been introduced, and a method of extrapolation using the similarity have been provided. 

Also, the descriptions of the compression and the expansion have been explained using the hs Mollier diagrams. Those diagrams, along with the schematic of the compressor and turbine stages, allows to describes with a decent accuracy the two previously mentioned transformation. 

The next section will be dedicated to the description of the combustion chamber and the main chemical equations will be defined.

 