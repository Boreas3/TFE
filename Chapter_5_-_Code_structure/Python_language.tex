\section{Python language}
\quad\, Python is a computing language that was created by Guido van Rossum at the Centrum Wiskunde \& Informatica (CWI - \url{https://www.cwi.nl}) in the early 1990s. Starting 1995, G. van Rossum continued to work on Python at the Corporation for National Research Initiatives (CNRI - \url{https://www.cnri.reston.va.us/}). Since the very first release, the language was open source. This means that the source code was accessible to anyone. 

From this time, Python progressively gained in popularity, and the community participating to the development of the software didn't stop to grow. Today, this language is used by many companies and for many types of applications. Indeed, this programming language is used for website creation, machine learning, automation, etc. 

Since Python is an open source language, it lives thanks to its community which creates and shares libraries. Indeed, what have already been implemented in the past can be freely used by the other users. Moreover, new users can easily start developing under Python thanks to huge amount of guide and documentation to starting learning about the Python language.

Also, Python is a programming language that allows the object oriented programming. This paradigm consist in the definition of blocks of code (called objects) which are able to interact together through relations. These relations are defined in order to solve a given problem.

The usage of Python within the scope of this work has been motivated by the previously mentioned characteristics.